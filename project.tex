\documentclass[12pt]{article}
\usepackage{amsfonts}        
\usepackage[T2A]{fontenc}    
\usepackage[utf8]{inputenc}  
\usepackage[russian]{babel}  
\usepackage{amsmath}
\usepackage{graphicx}        
\usepackage{lscape}
\usepackage{pdflscape}
\usepackage{hyperref}        
\usepackage{hyphenat}        
\usepackage{listings}        
\usepackage{color}
\usepackage{float}
\usepackage{comment}

\usepackage{rotating}
\usepackage{geometry}
 \geometry{
 a4paper,
 total={170mm,257mm},
 left=20mm,
 top=20mm,
 }


\graphicspath{./}

\definecolor{dkgreen}{rgb}{0,0.6,0}
\definecolor{gray}{rgb}{0.5,0.5,0.5}
\definecolor{mauve}{rgb}{0.58,0,0.82}

\newcommand{\bigcell}[2]{\begin{tabular}{@{}#1@{}}#2\end{tabular}}
\newcommand{\me}{\mathrm{e}}

\addto\captionsrussian{\renewcommand{\contentsname}{Зміст}}

\lstset{
  frame=none,
  language=Python,
  aboveskip=3mm,
  belowskip=3mm,
  showstringspaces=false,
  columns=flexible,
  basicstyle={\small\ttfamily},
  numbers=none,
  numberstyle=\tiny\color{gray},
  keywordstyle=\color{blue},
  commentstyle=\color{dkgreen},
  stringstyle=\color{mauve},
  breaklines=true,
  breakatwhitespace=true,
  tabsize=3
}

\setcounter{tocdepth}{3} %%where n is the level, starting with 0(chapters only)


\begin{document}

\begin{center}
\large{Одеський національний університет імені І.І.Мечникова
    \newline
Факультет математики, фізики та інформаційних технологій
    \newline
}
\bigbreak
\bigbreak
\bigbreak
\bigbreak
\bigbreak
\bigbreak
\bigbreak
\bigbreak
\\
\\
\\
\\
\bigbreak
\LARGE{ \textbf{Лабораторна робота}
\bigbreak
\textbf{Варіант 5}
\bigbreak

\bigbreak
}
\end{center}
\bigbreak
\bigbreak
\bigbreak
\bigbreak
\begin{flushright}
\large{Звіт студента ІІІ курсу
\bigbreak
денної форми навчання
\bigbreak
спеціальності
\bigbreak
113 Прикладна математика
\bigbreak
Жукова Павла Петровича
\bigbreak
\begin{center}

  \bigbreak
  \
  \bigbreak
  \
  \bigbreak
  \
  \bigbreak
  \
  \LARGE{ {}
  \
\bigbreak
\
\bigbreak
}
                  2020

\end{center}
}
\end{flushright}

\newpage

\section{Завдання №1}
1. Згрупувати початкові дані за незалежною ознакою. \\
2. Розрахувати коефіцієнт детермінації. \\
3. Розрахувати коефіцієнт Фехнера.\\
4. Сформулювати висновки.
\subsection{Розв'язання}
Розглядаються наступні показники виробничо-господарської діяльності 30 підприємств:
\\
\textbf{Y1} — Продуктивнiсть працi\\
\textbf{X7} — Фондовiддача\\ \\
Спочатку згрупуємо початкові дані за незалежною ознакою $X_{7} \;$. \\ Згруповані дані представимо у вигляді таблиці. \\
За даними таблиці розрахуємо коефіцієнт детермінації між $X_{7} \;$ і $Y_{1} \;$. \\
\begin{tabular}{ |c| c| c|} 
\hline
\bigcell{l}{Фондовiддача}
& 
\bigcell{l}{Продуктивнiсть працi} 
& 
\bigcell{l}{} \\
\hline
\bigcell{l}{$x_{1} = \dots = x_{7} = 1.3$}
&
\bigcell{l}{$\displaystyle y_{1} = 10.02, \;\; y_{2} = 9.42, \;\; y_{3} = 7.24, \\
y_{4} = 5.39, \;\; y_{5} = 6.57, \;\; y_{6} = 4.23, \\
y_{7} = 11.03, \;\; $}
&
\bigcell{l}{$\displaystyle \overline{y_{1}} = 7.7\\
\displaystyle\sum_{i = 1}^{7}(y_i - \overline{y})^2 = 38.295$}\\
\hline
\bigcell{l}{$x_{8} = \dots = x_{20} = 1.4$}
&
\bigcell{l}{$\displaystyle y_{8} = 9.37, \;\; y_{9} = 6.67, \;\; y_{10} = 5.68, \\
y_{11} = 5.22, \;\; y_{12} = 8.16, \;\; y_{13} = 6.48, \\
y_{14} = 8.77, \;\; y_{15} = 9.02, \;\; y_{16} = 6.7, \\
y_{17} = 6.69, \;\; y_{18} = 5.61, \;\; y_{19} = 5.59, \\
y_{20} = 18.0, \;\; $}
&
\bigcell{l}{$\displaystyle \overline{y_{2}} = 7.843\\
\displaystyle\sum_{i = 8}^{20}(y_i - \overline{y})^2 = 135.332$}\\
\hline
\bigcell{l}{$x_{21} = \dots = x_{28} = 1.6$}
&
\bigcell{l}{$\displaystyle y_{21} = 13.17, \;\; y_{22} = 3.78, \;\; y_{23} = 10.44, \\
y_{24} = 7.65, \;\; y_{25} = 7.0, \;\; y_{26} = 11.06, \\
y_{27} = 13.28, \;\; y_{28} = 9.27, \;\; $}
&
\bigcell{l}{$\displaystyle \overline{y_{3}} = 9.456\\
\displaystyle\sum_{i = 21}^{28}(y_i - \overline{y})^2 = 73.503$}\\
\hline
\bigcell{l}{$x_{29} = 1.7$}
&
\bigcell{l}{$\displaystyle y_{29} = 5.22, \;\; $}
&
\bigcell{l}{$\displaystyle \overline{y_{4}} = 5.22\\
\displaystyle\sum_{i = 29}^{29}(y_i - \overline{y})^2 = 0.0$}\\
\hline
\bigcell{l}{$x_{30} = 1.9$}
&
\bigcell{l}{$\displaystyle y_{30} = 6.54, \;\; $}
&
\bigcell{l}{$\displaystyle \overline{y_{5}} = 6.54\\
\displaystyle\sum_{i = 30}^{30}(y_i - \overline{y})^2 = 0.0$}\\
\hline
\end{tabular} \\ \\ \\ \\ 
Нехай нас цікавить ступінь тісноти статистичного зв’язку між результуючим показником y та пояснюючою змінною x. Очевидно, що ступінь тісноти зв’язку можна вважати максимальним, якщо по заданому значенню змінної x можна відтворити відповідне значення змінної y. І навпаки: якщо значення величини x не несе ніякої інформації про значення показника y , то зв’язок відсутній зовсім, і відповідний вимірник ступеня його тісноти повинен приймати мінімально можливе значення.\\
Введемо поняття ступеня виміру тісноти зв’язку під назвою коефіцієнт детермінації. Розглянемо його вибіркову характеристику. Обчислення вибіркового (емпіричного) значення коефіцієнта детермінації y по x виконується
за формулою 
Розрахуємо вибіркове значення дисперсії "нев'язок"  $\displaystyle \epsilon(x):$ згідно з формулою: $$\displaystyle S_{\epsilon}^2 = \frac{1}{n}\sum_{j=1}^s\sum_{i = 1}^{\nu_j}(y_{ji} - \overline{y_j})^2$$ отримаємо: 
$$\displaystyle S_{\epsilon}^2 = \frac{1}{30}(38.295 + 135.332 + 73.503 + 0.0 + 0.0) = 8.238$$
Далі розрахуємо $\displaystyle \overline{y} \; - $ середнє значення ознаки:
$$\displaystyle \overline{y} = \frac{1}{n}\sum_{i=1}^{n}y_{i} = \frac{1}{30}\sum_{i=1}^{30}y_{i} = 8.109$$
Тоді середнє значення ознаки буде дорівнювати:
$$\displaystyle S^{2}_{y} = \frac{\displaystyle \sum_{i=1}^{n} (y_{i} - \overline{y})^2}{n} = \frac{1}{30}\sum_{i=1}^{30} (y_{i} - \overline{y})^2 = 9.152$$
Маємо:
$$\displaystyle \widehat{K}_{d}(y;x) = 1 - \displaystyle\frac{\displaystyle S^{2}_{\epsilon}}{\displaystyle S^{2}_{y}} = 1 - \frac{8.238}{9.152} = 0.0999$$
\subsection{Висновок}
Коефіцієнт детермінації -  поняття ступеня виміру тісноти зв’язку, це статистичний показник, що використовується в статистичних моделях як міра залежності варіації залежної змінної від варіації незалежних змінних.  Вiн характеризує долю варiацiї залежної змiнної, обумовленої регресiєю. \\
Коефiцiєнт детермiнацiї належить промiжку $[0;1].$ Чим ближче значення коефіцієнта до 1, тим сильніше залежність і тим краще модель описує статистичні дані. 
У моєму випадку значення $K_{d} = 0.0999,$ тобто він вказує на те, що 9.99\% варіації рівняпродуктивності праці на досліджуваних підприємствах зумовлено варіацією фондовідаччі. При значеннях показників тісноти зв'язку менше $0.7$ величина коефіцієнта детермінації завжди буде нижче 50\%. Це означає, що на частку варіації факторних ознак доводиться менша частина в порівнянні з іншими неврахованими в моделі факторами, що впливають на зміну результативного показника. Побудовані при таких умовах регресивні моделі мають низьке практичне значення.
\subsection{Коефiцiєнт Фехнера}
Коефіцієнт Фехнера визначається виразом:
$$\displaystyle K_{\phi} = \frac{C - H}{n}$$
де $C, \; H$ - число випадків (спостережень, об’єктів), в яких по парі ознак $x$ і $y$ спостерігається відповідно збіг або незбіг знаків відхилень від середніх рівнів, а $n$ – кількість спостережень. \\
Зазначимо, що коефіцієнт Фехнера за абсолютною величиною не перевищує 1. Якщо він дорівнює $\pm 1$, то зв’язок між ознаками близький до функціонального. Про вигляд зв’язку – лінійний чи нелінійний – по величині коефіцієнта Фехнера нічого певного сказати не можна. \\
Визначити, чи існує зв’язок між фондовідаччею та продуктивністю праці, використовуючи дані таблиці. \\
\begin{table}[H]
\centering
\begin{tabular}{|c |c |c |c |c |c |c |c |c |}
\hline
\bigcell{l}{\\$i$}
& 
\bigcell{l}{$1$}
& 
\bigcell{l}{$2$}
& 
\bigcell{l}{$3$}
& 
\bigcell{l}{$4$}
& 
\bigcell{l}{$5$}
& 
\bigcell{l}{$6$}
& 
\bigcell{l}{$7$}
& 
\bigcell{l}{$8$}
\\
\hline
\bigcell{l}{Фондовiдачча $x$}
&
\bigcell{l}{$1.4$}
&
\bigcell{l}{$1.6$}
&
\bigcell{l}{$1.4$}
&
\bigcell{l}{$1.4$}
&
\bigcell{l}{$1.4$}
&
\bigcell{l}{$1.3$}
&
\bigcell{l}{$1.4$}
&
\bigcell{l}{$1.6$}
\\
\hline
\bigcell{l}{Знак \\ $x-\overline{x}$}
&
\bigcell{l}{$-$}
&
\bigcell{l}{$+$}
&
\bigcell{l}{$-$}
&
\bigcell{l}{$-$}
&
\bigcell{l}{$-$}
&
\bigcell{l}{$-$}
&
\bigcell{l}{$-$}
&
\bigcell{l}{$+$}
\\
\hline
\bigcell{l}{Продуктивнiсть працi $y$}
&
\bigcell{l}{$9.37$}
&
\bigcell{l}{$13.17$}
&
\bigcell{l}{$6.67$}
&
\bigcell{l}{$5.68$}
&
\bigcell{l}{$5.22$}
&
\bigcell{l}{$10.02$}
&
\bigcell{l}{$8.16$}
&
\bigcell{l}{$3.78$}
\\
\hline
\bigcell{l}{Знак \\ $y-\overline{y}$}
&
\bigcell{l}{$+$}
&
\bigcell{l}{$+$}
&
\bigcell{l}{$-$}
&
\bigcell{l}{$-$}
&
\bigcell{l}{$-$}
&
\bigcell{l}{$+$}
&
\bigcell{l}{$+$}
&
\bigcell{l}{$-$}
\\
\hline
\end{tabular}
\end{table}
\begin{table}[H]
\centering
\begin{tabular}{|c |c |c |c |c |c |c |c |c |}
\hline
\bigcell{l}{\\$i$}
& 
\bigcell{l}{$9$}
& 
\bigcell{l}{$10$}
& 
\bigcell{l}{$11$}
& 
\bigcell{l}{$12$}
& 
\bigcell{l}{$13$}
& 
\bigcell{l}{$14$}
& 
\bigcell{l}{$15$}
& 
\bigcell{l}{$16$}
\\
\hline
\bigcell{l}{Фондовiдачча $x$}
&
\bigcell{l}{$1.4$}
&
\bigcell{l}{$1.6$}
&
\bigcell{l}{$1.6$}
&
\bigcell{l}{$1.4$}
&
\bigcell{l}{$1.6$}
&
\bigcell{l}{$1.6$}
&
\bigcell{l}{$1.4$}
&
\bigcell{l}{$1.6$}
\\
\hline
\bigcell{l}{Знак \\ $x-\overline{x}$}
&
\bigcell{l}{$-$}
&
\bigcell{l}{$+$}
&
\bigcell{l}{$+$}
&
\bigcell{l}{$-$}
&
\bigcell{l}{$+$}
&
\bigcell{l}{$+$}
&
\bigcell{l}{$-$}
&
\bigcell{l}{$+$}
\\
\hline
\bigcell{l}{Продуктивнiсть працi $y$}
&
\bigcell{l}{$6.48$}
&
\bigcell{l}{$10.44$}
&
\bigcell{l}{$7.65$}
&
\bigcell{l}{$8.77$}
&
\bigcell{l}{$7.0$}
&
\bigcell{l}{$11.06$}
&
\bigcell{l}{$9.02$}
&
\bigcell{l}{$13.28$}
\\
\hline
\bigcell{l}{Знак \\ $y-\overline{y}$}
&
\bigcell{l}{$-$}
&
\bigcell{l}{$+$}
&
\bigcell{l}{$-$}
&
\bigcell{l}{$+$}
&
\bigcell{l}{$-$}
&
\bigcell{l}{$+$}
&
\bigcell{l}{$+$}
&
\bigcell{l}{$+$}
\\
\hline
\end{tabular}
\end{table}
\begin{table}[H]
\centering
\begin{tabular}{|c |c |c |c |c |c |c |c |c |}
\hline
\bigcell{l}{\\$i$}
& 
\bigcell{l}{$17$}
& 
\bigcell{l}{$18$}
& 
\bigcell{l}{$19$}
& 
\bigcell{l}{$20$}
& 
\bigcell{l}{$21$}
& 
\bigcell{l}{$22$}
& 
\bigcell{l}{$23$}
& 
\bigcell{l}{$24$}
\\
\hline
\bigcell{l}{Фондовiдачча $x$}
&
\bigcell{l}{$1.6$}
&
\bigcell{l}{$1.4$}
&
\bigcell{l}{$1.4$}
&
\bigcell{l}{$1.3$}
&
\bigcell{l}{$1.3$}
&
\bigcell{l}{$1.3$}
&
\bigcell{l}{$1.4$}
&
\bigcell{l}{$1.4$}
\\
\hline
\bigcell{l}{Знак \\ $x-\overline{x}$}
&
\bigcell{l}{$+$}
&
\bigcell{l}{$-$}
&
\bigcell{l}{$-$}
&
\bigcell{l}{$-$}
&
\bigcell{l}{$-$}
&
\bigcell{l}{$-$}
&
\bigcell{l}{$-$}
&
\bigcell{l}{$-$}
\\
\hline
\bigcell{l}{Продуктивнiсть працi $y$}
&
\bigcell{l}{$9.27$}
&
\bigcell{l}{$6.7$}
&
\bigcell{l}{$6.69$}
&
\bigcell{l}{$9.42$}
&
\bigcell{l}{$7.24$}
&
\bigcell{l}{$5.39$}
&
\bigcell{l}{$5.61$}
&
\bigcell{l}{$5.59$}
\\
\hline
\bigcell{l}{Знак \\ $y-\overline{y}$}
&
\bigcell{l}{$+$}
&
\bigcell{l}{$-$}
&
\bigcell{l}{$-$}
&
\bigcell{l}{$+$}
&
\bigcell{l}{$-$}
&
\bigcell{l}{$-$}
&
\bigcell{l}{$-$}
&
\bigcell{l}{$-$}
\\
\hline
\end{tabular}
\end{table}
\begin{table}[H]
\centering
\begin{tabular}{|c |c |c |c |c |c |c |c |c |}
\hline
\bigcell{l}{\\$i$}
& 
\bigcell{l}{$25$}
& 
\bigcell{l}{$26$}
& 
\bigcell{l}{$27$}
& 
\bigcell{l}{$28$}
& 
\bigcell{l}{$29$}
& 
\bigcell{l}{$30$}
& 
\bigcell{l}{$$\sum$$}
& 
\bigcell{l}{$$\overline{z}$$}
\\
\hline
\bigcell{l}{Фондовiдачча $x$}
&
\bigcell{l}{$1.3$}
&
\bigcell{l}{$1.9$}
&
\bigcell{l}{$1.3$}
&
\bigcell{l}{$1.7$}
&
\bigcell{l}{$1.4$}
&
\bigcell{l}{$1.3$}
&
\bigcell{l}{$43.7$}
&
\bigcell{l}{$1.457$}
\\
\hline
\bigcell{l}{Знак \\ $x-\overline{x}$}
&
\bigcell{l}{$-$}
&
\bigcell{l}{$+$}
&
\bigcell{l}{$-$}
&
\bigcell{l}{$+$}
&
\bigcell{l}{$-$}
&
\bigcell{l}{$-$}
&
\bigcell{l}{}
&
\bigcell{l}{}
\\
\hline
\bigcell{l}{Продуктивнiсть працi $y$}
&
\bigcell{l}{$6.57$}
&
\bigcell{l}{$6.54$}
&
\bigcell{l}{$4.23$}
&
\bigcell{l}{$5.22$}
&
\bigcell{l}{$18.0$}
&
\bigcell{l}{$11.03$}
&
\bigcell{l}{$243.27$}
&
\bigcell{l}{$8.109$}
\\
\hline
\bigcell{l}{Знак \\ $y-\overline{y}$}
&
\bigcell{l}{$-$}
&
\bigcell{l}{$-$}
&
\bigcell{l}{$-$}
&
\bigcell{l}{$-$}
&
\bigcell{l}{$+$}
&
\bigcell{l}{$+$}
&
\bigcell{l}{}
&
\bigcell{l}{}
\\
\hline
\end{tabular}
\end{table}
Маємо  $C = 17, \; \; H = 13,$ тоді
$$\displaystyle K_{\phi} = \frac{C - H}{n} = 0.13$$
\end{document}
